\documentclass[10pt]{beamer}

  % Math Packages
  \usepackage{algorithmic}
  \usepackage{amsmath}
  \usepackage{amsthm}
  \usepackage{mathtools}

  % Graphics Packages
  \usepackage[export]{adjustbox}
  \usepackage{graphicx}
  \usepackage{pgf}

  % Other packages
  \usepackage{color}
  \usepackage{csquotes}  % Adds ability to use display quotes
  \usepackage{listings}

  % Theme Settings
  \usetheme{metropolis}
  \usefonttheme{professionalfonts}

  % Special commands
  \DeclareMathOperator*{\argmin}{arg\,min}
  \DeclareMathOperator*{\argmax}{arg\,max}

\title{Global Solutions and Income Fluctuation Problem}
\author{Chase Coleman}
\institute{NYU Stern}
\date[]{\today}

\begin{document}

% Title Slide
\begin{frame}
  \thispagestyle{empty}
  \titlepage
\end{frame}


% --------------------------------------- %
% Introduction
% --------------------------------------- %
\section{Introduction to Income Fluctuation Problem}

\begin{frame} \frametitle{Income Fluctuation Problem (IFP)}

  Foundation for modern heterogeneous agent macro

  \vspace{0.5cm}

  Partial equilibrium model of a single consumer's problem
  \begin{itemize}
    \item Wages exogenously given
    \item Asset return exogenously given
  \end{itemize}

\end{frame}

\begin{frame} \frametitle{Income Fluctuation Problem (IFP)}

  Single consumer with preferences $$\max_{\{c_t(s^t)\}_{t=0}^{\infty}} E_0 \left[ \sum_{t=0} \beta^t u(c_t(s^t)) \right]$$

  Markov stochastic process, $\{ s_t \}$

  Receives stochastic income each period, $y(s_t)$

  Income can either be
  \begin{itemize}
    \item Consumed, $c_t(s^t)$ or
    \item Saved in risk-free asset, $a_t(s^t)$, with return $r$
  \end{itemize}

\end{frame}

\begin{frame} \frametitle{Recursive Formulation}

  Problem can be ``Bellmanized'' with $(a_t, s_t)$ as the state variables

  \begin{align*}
    V(a_t, s_t) &= \max_{a_{t+1}} u(c_t) + \beta E \left[ V(a_{t+1}, s_{t+1}) | s_t \right] \\
              &\text{Subject to} \\
              &c_t = (1 + r) a_t + y(s_t) - a_{t+1}
  \end{align*}

\end{frame}

\section{Solving the IFP}

\begin{frame} \frametitle{Solving the IFP}
  What does it mean to ``solve'' the IFP?

  \vspace{0.5cm}

  Want to characterize
  \begin{itemize}
    \item Policy function: $a^*(a_t, s_t)$
    \item Value function: $V(a_t, s_t)$
  \end{itemize}

\end{frame}

\begin{frame} \frametitle{Three Iterative Algorithms}

  Will introduce three global solution methods.

  All three are variants on value function iteration

  \begin{enumerate}
    \item Discretized Value Function Iteration
    \item Interpolated Value Function Iteration
    \item Envelope Condition Method
  \end{enumerate}

\end{frame}

\begin{frame} \frametitle{Discretized Value Function Iteration}

  \begin{enumerate}
    \item Choose grids on state variables: $\mathcal{A} \times \mathcal{S}$
    \item Initial guess of value function: $V_0(a, s) \;\forall (a, s) \in \mathcal{A} \times \mathcal{S}$
    \item Given $V_j$, for each $(a_t, s_t) \in \mathcal{A} \times \mathcal{S}$ \label{updatestep}
      \begin{enumerate}
        \item Find $a_{t+1} \in \mathcal{A}$\footnote{Just evaluate at each grid point and choose the max. Very computationally inexpensive} such that
          \begin{align*}
            a_{t+1} \in \argmax u(y(s_t) + (1+r) a_t - a_{t+1}) + \beta E \left[ V_j(a_{t+1}, s_{t+1}) \right]
          \end{align*}
        \item Update policy function $$a^*(a_t, s_t) = a_{t+1}$$
        \item Update value function $$V_{j+1}(a_t, s_t) = u(c_t) + \beta E \left[V_{j}(a_{t+1}, s_{t+1}) \right]$$
      \end{enumerate}
    \item If $d(V_{j+1}, V_j) < \varepsilon$ then done, otherwise return to \ref{updatestep}
  \end{enumerate}

\end{frame}

\begin{frame} \frametitle{Interpolated Value Function Iteration}

  \begin{enumerate}
    \item Choose grids on state variables: $\mathcal{A} \times \mathcal{S}$
    \item Initial guess of value function: $V_0(a, s) \;\forall (a, s) \in \mathcal{A} \times \mathcal{S}$
    \item Create functional approximation of $V_j$ using interpolation routines: $\hat{V}_j$ \label{interpstep}
    \item Given $\hat{V}_j$, for each $(a_t, s_t) \in \mathcal{A} \times \mathcal{S}$
      \begin{enumerate}
        \item Using optimization routines, find
          \begin{align*}
            a_{t+1} \in \argmax u(y(s_t) + (1+r) a_t - a_{t+1}) + \beta E \left[ \hat{V}_j(a_{t+1}, s_{t+1}) \right]
          \end{align*}
        \item Update policy function $$a^*(a_t, s_t) = a_{t+1}$$
        \item Update value function $$V_{j+1}(a_t, s_t) = u(c_t) + \beta E \left[\hat{V}_{j}(a_{t+1}, s_{t+1}) \right]$$
      \end{enumerate}
    \item If $d(V_{j+1}, V_j) < \varepsilon$ then done, otherwise return to \ref{interpstep}
  \end{enumerate}

\end{frame}

\begin{frame} \frametitle{Envelope Condition Method}

  \begin{enumerate}
    \item Choose grids on state variables: $\mathcal{A} \times \mathcal{S}$
    \item Initial guess of value function: $V_0(a, s) \;\forall (a, s) \in \mathcal{A} \times \mathcal{S}$
    \item Create functional approximation of $V_j$ using interpolation routines: $\hat{V}_j$ \label{interpstep}
    \item Given $\hat{V}_j$, for each $(a_t, s_t) \in \mathcal{A} \times \mathcal{S}$
      \begin{enumerate}
        \item Use envelope condition to get $c_t^* = (u')^{-1} \left( \frac{\partial \hat{V}_j(a_t, s_t) / \partial a_t}{1 + r} \right)$
        \item Update policy function $$a^*(a_t, s_t) = y(s_t) + (1+r) a_t - c_t^*$$
        \item Update value function $$V_{j+1}(a_t, s_t) = u(c^*_t) + \beta E \left[\hat{V}_{j}(a_{t+1}, s_{t+1}) \right]$$
      \end{enumerate}
    \item If $d(V_{j+1}, V_j) < \varepsilon$ then done, otherwise return to \ref{interpstep}
  \end{enumerate}

\end{frame}

\begin{frame} \frametitle{Comparisons}

  Thoughts on different algorithms:
  \begin{itemize}
    \item Discretized VFI:
      \begin{itemize}
        \item Pros: Easy to code
        \item Cons: Infeasible for big problems
      \end{itemize}
    \item Interpolated VFI:
      \begin{itemize}
        \item Pros: Very general, solves slightly bigger problems
        \item Cons: Optimization routines are a drag in terms of speed, infeasible for big problems
      \end{itemize}
    \item Envelope Condition Method:
      \begin{itemize}
        \item Pros: Very fast, frequently can be used in some other models to get a ``closed form'' for certain choice variables
        \item Cons: Careful that derivatives are sufficiently nice, does not work for all models
      \end{itemize}
  \end{itemize}

\end{frame}

\section{Example}

\begin{frame} \frametitle{Example}

  Will walk through example of discretized value function together

  \vspace{0.5cm}

  Practice problem will require solving a perturbation (or two) of the income fluctuation problem using other algorithms

\end{frame}

\end{document}

